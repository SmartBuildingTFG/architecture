\documentclass[12pt, a4paper, twoside]{article}

%% Preamble
\usepackage{umatfgspanish}
\graphicspath{{./images/}}

\begin{document}

\includepdf[noautoscale=true, width=\paperwidth]{title.pdf}

\newpage

\tableofcontents

%% Sections
\section{Proceso de Análisis del Negocio}
\subsection{Preparación para el Análisis del Negocio}
\subsubsection{Contextualización}
El contexto que se va a desarrollar a continuación parte del concepto de Internet de las Cosas (IoT).

Este concepto se basa en considerar internet usando un conjunto de usuarios más amplio que el 
habitual, no solo incluye a las personas y a servidores, sino también trata a cualquier objeto 
físico o animal como un candidato a formar parte del sistema.

La consecuencia
Los resultados derivados
El potencial de

Internet, consiste en una red de usuarios y servidores conectadas entre sí.
Sin embargo, existen conceptos más amplios que abarcan un conjunto de elementos todavía más ambicioso,
como es el de Internet de las Cosas (IoT), que, además de los anteriores, incluye a cualquier objeto físico,
 animal...

Desde que se empezó a aplicar el concepto de IoT, ha ido creciendo el números de <<cosas>> conectadas a él. 
Por ejemplo...
En los últimos años, el uso del IoT ha aumentado exponencialmente [ENSEÑAME LO QUE TIENES].
\subsubsection{Problemas y oportunidades identificadas}
REF: 
Estevez E., Pardo, T., & Scholl, J. (2021). Smart cities and smart governance : towards the 22nd century sustainable city. Springer.



Se puede identificar una oportunidad en el hecho de que hay muchos campos de aplicación en los que se
puede aplicar el IoT y todavía no se ha hecho uso de.
El IoT es una tecnología que lleva poco tiempo en auge y son muchos los campos en los que se puede 
aplicar para sacar ventaja, obtener beneficios, mejorar la calidad...

Se puede identificar un problema en [LAS COSAS ESTAS DE CIUDADES]

El potencial que se le puede sacar a esta red de dispositivos es muy elevado. Existen conceptos tales como:
 - IoT Orchestation
 - Fog Computing
 - 
\end{document}