\documentclass[12pt, a4paper, twoside]{article}

%% Preamble
\usepackage{umatfgspanish}
\usepackage{tabto}
\usepackage{hyperref}
\usepackage{makecell}
\usepackage{longtable}
\hypersetup{
  colorlinks=true,
  linkcolor=blue,
  filecolor=magenta,      
  urlcolor=cyan,
  pdftitle={Overleaf Example},
  pdfpagemode=FullScreen,
}
\newcommand\ttab{\tab \hspace{-5cm}}
\graphicspath{{./images/}}

\begin{document}
\includepdf[noautoscale=true, width=\paperwidth]{title.pdf}
\newpage
\tableofcontents

%% Sections
\section{Introducción}
\subsection{Propósito}
Desarrollar una plataforma virtual común que permita:
\begin{itemize}
  \item Gestionar y organizar la información de cualquier edificio 
        dentro de una ciudad.
  \item Interconectar cualquier elemento integrable a la red de Internet 
        (dispositivos IoT) que se encuentren en un edificio (Smart Building)
        usando interfaces de comunicación que ya son estándar.
\end{itemize}

\subsection{Ámbito}
Debido a las características del software que hacen uso de las tecnologías 
``Internet de las Cosas'' (IoT) para la gestión de dispositivos conectados a
la red de un Edificio Inteligente (Smart Building), se ha decidido darle el nombre de
``IBuilding''.

% META
IBuilding pretende ser un gestor de los edificios de una ciudad con el que
poder consultar información sobre ellos y manipular sus dispositivos.
 
 % OBJETIVO
Para ello, IBuilding deberá almacenar datos de los edificios y 
deberá disponer de herramientas para acceder y manipular los dispositivos que se instalen
en los mismos.

% Beneficio
De esta manera, se puede enriquecer el valor de una ciudad, permitiendo digitalizar más
datos sobre ella y poder generar una información más fiel y útil para que otras personas
y servicios digitales puedan hacer uso de ella.

IBuilding dispondrá de un servidor central (llamado DataBuilding) que, usando la tecnología
de Fiware, almacenará los datos propios de cada edificio (como la ubicación o el 
tipo de servicio que ofrecen) y los dispositivos disponibles en el mismo. 
Se podrán consultar los datos a través de una API que usará modelos estándares 
de datos.

Los diferentes dispositivos necesitarán controladores que permitan conectarse
a internet para poder manipularlos mediante algún API.

Como ejemplo se desarrollarán algunos Controllers, para hacer uso y alimentar a la API.
Estos dotarán a los dispostivos (Ya sean físicos o virtuales) de acceso a internet, usando interfaces 
estándares, como UltraLight 2.0, OneM2M o directamente NGSI-LD. Ejemplos de Controllers podrían
ser SensorController o ANNPlateController.

El desarrollo de una interfaz de usuario puede ser útil para la gestión de la plataforma
a nivel usuario. Es por eso que también habrá un frontal para hacer uso de IBuilding
que se conectará al DataBuilding y podrá visualizar/modificar los datos oportunos.
Para referirnos a este frontal usaremos el nombre de FrontalBuilding.

\section{Visión general del producto}
\subsubsection{Perspectiva del Producto}
Desde el punto de visto externo, IBuilding puede ser visto como un servicio de integración que
\begin{itemize}
  \item ofrezca datos sobre los edificio a aplicaciones terceras.
  \item permita controlar dispositivos mediante una API estándar y conocida.
  \item permita conectar dispositivos IoT de la Smart Building a la red.
\end{itemize}

Para la implementación de esta interfaz Fiware es un candidato ideal, ya que ofrece NGSI-LD,
una interfaz de comunicación reconocida por diferentes organismo importantes de referencia en toda la Unión Europea.

Además, Fiware deja abierta la posiblidad de implementar otras interfaces que se usan en desarrollo de dispositivos
IoT como UltraLight 2.0 o OneM2M, que también se considera un estándar reconocido por organismos
como la ETSI.

\begin{figure}[h]
  \centering
  \includegraphics[width=0.8\textwidth]{IBuildingGenericComponents.1.1.png}
  \caption{Niveles de integración posibles siguiendo una aproximación con Gemelo Digital}
\end{figure}

\subsubsection{Interfaces de Usuario}
Se utilizará como guía de diseño Material Design.

Material Design es un ecosistema de diseños desarrollada por Google
que siguen un conjunto de reglas para definir el estilo de la aplicación.

Debido a las características de los usuarios que hacen uso habitual de internet,
aplicar estilos de diseño similares a las de aplicaciones populares puede ayudar a aumentar
el nivel de familiarización e intuitividad de la aplicación (Twitter, Facebook, Google...).

Muchas de las aplicaciones populares hacen uso de esta guía.
Material Design tiene unas especificaciones muy extensas y muy bien definidas. 

\subsubsection{Interfaces Software}
Para poder afirmar que un sistema está impulsado por Fiware, es necesario incluir un Context Broker del catálogo de Fiware.
Se ha elegido Orion-LD Context Broker, por implementar la interfaz NGSI-LD y por sus características de tamaño.

Además, por cuestiones de Seguridad, también se incluirán IdM Keyrock y Wilma, que son softwares completamente integrables y
compatibles con NGSI-LD. 

También se incluirá FogFlow como solución para orquestar flujos de trabajos que involucren a los dispositivos IoT. Esta solución 
se beneficia de la computación en la niebla para optimizar llamadas innecesarias a servidores y ahorrar tiempo y dinero por ahorro
de cómputos en servidores en la nube. Este servicio también tiene compatibilidad con NGSI-LD.

Como último, se incluirán los IoT Agents: Softwares de Fiware que se encargan de adaptar la interfaz NGSI-LD a otros protocolos
e interfaces para poder ampliar las interfaces de comunicación que se ofrecen. Los IoT Agents que se van a incluir en este proyecto
son IoT Agent - Ultralight 2.0 y OpenMTC.
\begin{table}
  \footnotesize
  \begin{tabular}{ |l|c|c|l| } 
   \hline
   Nombre                                                                        & Abreviatura & Última Versión & Interfaz Relevante \\ \hline
   \href{https://github.com/FIWARE/context.Orion-LD}{Orion-LD Context Broker}     & OCB         & 1.1.2          & NGSI-LD \\ \hline 
   \href{https://github.com/ging/fiware-idm}{Identity Manager - Keyrock}          & IdM Keyrock & 8.3.1          & IdM GE API \\ \hline
   \href{https://github.com/ging/fiware-pep-proxy}{PEP Proxy - Wilma}             & Wilma       & 8.3            & NGSI-LD \\ \hline
   \href{https://github.com/smartfog/fogflow}{FogFlow}                            & FogFlow     & 3.2.8          & NGSI-LD \\ \hline
   \href{https://github.com/telefonicaid/iotagent-ul}{IoT Agent - Ultralight 2.0} & IotAgentUL  & 1.23.0         & Ultralight 2.0 \\ \hline
   \href{https://github.com/OpenMTC/OpenMTC}{OpenMTC}                             & OpenMTC     & 1.3.0          & OneM2M \\ \hline
  \end{tabular}
\end{table}

\paragraph{NGSI-LD}
La interfaz NGSI-LD es muy apropiada para el proyecto. Está estandarizada por la ETSI y
son muchas las aplicaciones de ámbitos inteligentes que lo utilizan.

Tiene una base semántica muy bien definida sobre RDF y Semática Web, e implementa el Linked Data, 
que permite interconectar los recursos por toda la web, por lo que le da más calidad y usabilidad
a los datos.
\paragraph{IdM GE API}
La seguridad en la IoT es un asunto muy crítico, ya que son muchos procesos que se ejecutan automáticamente
desde diversos microservicios y dipositivos que necesitan validar su autenticidad.

Por este motivo se pretende utilizar Identity Manager, que implementa protocolos de seguridad
actuales, como OAuth2.0 para manipular los datos de los usuarios y es compatible con Fiware.
Esta interfaz se utilizará para gestionar la seguridad del sistema.
Se puede consultar su documentación desde los siguientes enlaces: 

https://keyrock.docs.apiary.io/
https://keyrock-fiware.github.io
\paragraph{Ultralight 2.0}
Protocolo orientado a la comunicación entre dispositivos IoT.
Sus características hacen que sea muy ligero y óptimo para el IoT, sobretodo
en escenarios de recursos muy limitados.
Se utilizará para ampliar las interfaces que ofrece el sistema al mundo exterior
para incrementar las opciones de integración con otros dispositivos.

https://fiware-iotagent-ul.readthedocs.io/en/latest/index.html

\paragraph{OneM2M}
Es una organización cuyo objetivo es el de crear un estandar internacional para la interoperabilidad
entre dispositivos físicos e internet.

Con este estándar se pueden orquestar dispositivos IoT de una manera muy eficiente y óptima, además,
también está reconocido por diversos organismos como la ETSII, por lo tanto también es interesante
implementarlo, por su nivel de internacionalización y su verdadera utilidad en la IoT.

\section{Funciones del producto}
Cualquier persona podrá:
 Registrarse y loguearse en la aplicación y podrá visualizar un mapa con los edificios cercanos.
 Listar los edificios con diferentes filtros (tipos de edificio, por distancia, ...)
 Visualizar la información del edificio si tiene permisos.

Creación y configuración del Edificio:
 Un usuario podrá crear edificios. Para ello rellenará un formulario con los datos necesarios que
 tendrán que ser revisado por un adminsitrador del sistema.
 Una vez creado un edificio se podrá gestionar la información del mismo y añadir dispositivos IoT oportunos.

Roles en un edificio:
 Usuarios tendrán roles dentro de un edificio. La información y el uso de los dispositivos a los cuales
 los usuarios pueden acceder se puede configurar por roles.

\begin{figure}[h]
  \centering
  \includegraphics[width=0.8\textwidth]{UserCase.1.0.png}
  \caption{Diagrama de casos de uso}
\end{figure}

 \subsubsection{Características de los Usuarios}
 El usuario objetivo es aquel que vive en ciudades grandes con un nivel tecnológico avanzado,
 por lo tanto, es un usuario familirializado con las tecnologías y que hace un uso
 habitual de internet, en su mayoría.

 El rango incluye usuarios de todas las edades y niveles educativos, que, por lo general
 no tienen un nivel técnico muy avanzado.

 También podemos encontrarnos con un conjunto de usuarios que
 no estén tan habituadas a usar ciertas aplicaciones o que tengan algún nivel de discapacidad
 que dificulte su utilización.

\subsection{Definiciones}
\begin{itemize}
    \item \textbf{Sensores}\ttab Los sensores son dispositivos IoT cuya labor principal es recolectar datos.
    \item \textbf{Actuadores}\ttab Los actuadores son dispositivos IoT cuya función principal hacer una tarea.
\end{itemize}

\section{Referencias}

\section{Requisitos}
\subsection{Requisitos funcionales}
{\footnotesize
\begin{longtable}{ |c|c|l| }
  \hline
  ID      & Título & Descripción \\ \hline
  RF-EDI1 & RUD edificios & \makecell[l]{Se pueden manejar los datos de un edificio: \\
    Cualquier usuario podrá visualizar los datos de edificios \\
    validados. Los editores y el usuario que envió la solicitud\\
    de creación del edificio (creador) podrá editar los datos. \\
    El creador del edificio o cualquier administrador de la \\
    plataforma será el único que podrá borrarlo.
    } \\ \hline
  % NF: Como lo ordeno, filtros, buscador... Detalles de la vista de detalles del edificio (mapa, salseo...)
  RF-EDI2 & \makecell{Solicitud de \\ creación de edificio}
    & \makecell[l]{Cualquier usuario logueado puede solicitar la creación\\
    de un edificio rellenando un formulario. Al enviar los datos \\
    se insertan en la base de datos con un atributo que indique \\
    su estado, quedando solamente visibles para los
    \\ administradores. 
    } \\ \hline
  RF-EDI3 & \makecell{Aceptar solicitud \\ creación de edificio}
    & \makecell[l]{Un administrador podrá aceptar solicitudes de creación de \\
    un edificio. Una vez aceptadas cambia el estado a aprobado \\
    y pasan a ser visibles para todos los usuarios.
    }\\ \hline
  RF-EDI4 & \makecell{Denegar solicitud \\ creación de edificio}
    & \makecell[l]{Un administrador podrá denegar solicitudes de creación de \\
    un edificio. Cuando se rechazen, estos se eliminarán de la \\
    base de datos. 
    } \\ \hline
  
  RF-USU1 & RUD de Usuarios 
  & \makecell[l]{Los usuarios logueados podrán visualizar los perfiles \\
  de otras personas, actualizar sus datos o darse de baja. 
  }\\ \hline
  % Voy a utilizar (no me acuerdo como se llama) La cosa esta de auth de fiware.
  RF-USU2 & Registro en la aplicación 
  & Cualquier usuario podrá registrarse en el sistema. 
  \\ \hline 
  RF-USU3 & Validación del acceso 
  & \makecell[l]{Cualquier usuario registrado podrá iniciar sesión \\
  en la aplicación usando las credenciales proporcionadas \\
  en el registro.
   } \\ \hline
  % NF: CUALQUIER USUARIO REGISTRADO O NO PODRÁ ACCEDER A LA APLICACIÓN. UN USUARIO NO REGISTRADO ENTRARÁ DE FORMA ANÓNIMA.
  RF-USU4 & Cerrar sesión & Cualquier usuario logueado podrá cerrar sesión. \\ \hline
  % NF: EL TOKEN GENERADO SE GUARDARÁ EN SESSION STGORATE... USARE JSONWEWTOKEN... DURACIÓN DE SESIÓN? 24H? ES PREGUNTA.


  RF-ROL1 & Otorgar rol de edficio 
  & \makecell[l]{El creador de un edificio podrá otorgar roles en su edificio \\
  a cualquier usuario registrado en la aplicación. El criterio \\
  para otorgar los roles queda a cargo del creador.
  } \\ \hline
 
  RF-INI1 & Acceso a la página de inicio & Cualquier usuario puede acceder a la página principal. \\ \hline
 
  RF-DIS1 & CRUD dispositivos 
  & \makecell[l]{Dentro de un edificio se podrán incluir dipositivos IoT que \\
    interactúen con otros usuarios, entendiendo por\\
    interacción la lectura de datos o activarlos para\\
    que realicen alguna función.\\
    El creador o los editores de un edificio podrán \\ 
    añadir, editar o eliminar nuevos dispositivos al mismo. \\
    Podrán indicar la visibilidad de los datos: \\
    público: Cualquier usuario que acceda al edificio podrá \\
    interactuar con el dispositivo. \\
    privado: Solo los usuarios que tengan uno de los siguientes\\ 
    roles: creador, editor o miembro, podrán interactuar con \\
    el dispositivo.
  } \\ \hline
\end{longtable}
}
{\footnotesize
 \subsection{Requisitos no funcionales}
 \begin{longtable}{ |c|c|l| }
  \hline
  ID       & Título & Descripción \\ \hline
  RNF-EDI1 & \makecell{Ordenación en el \\ listado de edificios}
  & \makecell[l]{Los edificios se ordenarán por cercanía \\
  a la ubicación de la persona si se pudiera utilizar. \\
  En caso de no estar disponible, se ordenará por orden \\
  alfabético. 
  }\\ \hline

  RNF-EDI2 & \makecell{
    Filtrado en el listado \\
    de ordenación de edificios}
  & \makecell[l]{Los edificios se podrán filtrar por nombre, \\
  tipo de edificio, distancia o si tiene algún rol especial \\
  en ese edificio.
  } \\ \hline

  RNF-EDI3 & \makecell{Mapa en los \\ detalles de un edificio}
  & \makecell[l]{En la vista de los detalles, se incluirá  \\
  un mapa mostrando la ubicación del edificio.\\
  Utilizará el API de Google Maps.
   } \\ \hline
  % NF: Como lo ordeno, filtros, buscador... Detalles de la vista de detalles del edificio (mapa, salseo...)

  RNF-USU1 & \makecell{Servicio de \\Autenticación de Fiware}
  & \makecell[l]{Se utilizarán los microservicios de \\
  autenticación de Identity Manager para la \\
  autenticación de los usuarios y de los dispositivos \\
  en el sistema por compatibilidad con Fiware.
  } \\ \hline
  % Voy a utilizar (no me acuerdo como se llama) La cosa esta de auth de fiware.
  
  RNF-USU2 & Acceso anónimo 
  & \makecell[l]{Un usuario no registrado podrá entrar en \\
   la aplicación de forma anónima. Solo podrán\\
   visualizar datos públicos de los edificios.
   } \\ \hline
  % NF: CUALQUIER USUARIO REGISTRADO O NO PODRÁ ACCEDER A LA APLICACIÓN. UN USUARIO NO REGISTRADO ENTRARÁ DE FORMA ANÓNIMA.
  
  RNF-USU3 & Token de sesión 
  & \makecell[l]{Se utilizará un token de sesión JWT \\
  (Json Web Token) que expirará cada 24 horas y que la \\
  aplicación guardará en el session storage.
   } \\ \hline
   
  RNF-USU4 & Almacenamiento de la constraseña 
  & \makecell[l]{Se almacenarán las constraseñas de los \\
  usuarios en una base de datos dedicada a las \\
  credenciales gestionada por Identity Manager
  } \\ \hline 
  % NF: EL TOKEN GENERADO SE GUARDARÁ EN SESSION STGORATE... USARE JSONWEWTOKEN... DURACIÓN DE SESIÓN? 24H? ES PREGUNTA.
  RNF-ROL1 & Maestro roles de la aplicación 
  & \makecell[l]{Hay 3 roles en la aplicación: \\
  administrador, usuario y anónimo.
  } \\ \hline
  RNF-ROL2 & Maestro roles de un edificio 
  & \makecell[l]{Cualquier usuario registrado podrá tener \\
  1 rol de los siguientes en cada edificio:  creador, editor,\\
  miembro, público
  } \\ \hline

  % Requisitos No funcionales Fiware
  RNF-FIW1 & Capacidad de recopilación de datos 
  & \makecell[l]{Debe ser capaz de recopilar datos \\
  de los sensores y dispositivos ubicados en diferentes áreas \\
  del edificio y en tiempo real. Además, debe ser capaz de\\
  manejar grandes volúmenes de datos.
  } \\ \hline
  RNF-FIW2 & Integración con sistemas de terceros 
  & \makecell[l]{Debe ser compatible con sistemas de terceros\\
  para garantizar la interoperabilidad. Debe poder integrarse\\
  sin problemas con sistemas existentes en el edificio.
  } \\ \hline
  RNF-FIW3 & Análisis de datos
  & \makecell[l]{Debe ser capaz de analizar los datos\\
  recopilados de los sensores y dispositivos para proporcionar\\
  información valiosa.
  } \\ \hline

  RNF-FIW4 & Control y gestión de dispositivos
  & \makecell[l]{
    Debe ser capaz de controlar y gestionar dispositivos\\
    y sistemas conectados en el edificio, como sistemas de\\
    iluminación, HVAC, puertas, persianas, entre otros. Esto\\
    debe ser posible desde una única plataforma central.
  } \\ \hline

  RNF-FIW5 & Seguridad y privacidad de los datos
  & \makecell[l]{
    Los datos sensibles se tratarán de manera especial para aumentar su seguridad.
    Por ejemplo, cifrar los datos o aplicarle un hash según sea necesario, como 
    matrículas de coches.
  } \\ \hline

  RNF-FIW6 & Escalabilidad
  & \makecell[l]{
    Debe ser escalable para que pueda manejar el\\
    crecimiento de edificios inteligentes.\\
    Debe ser capaz de adaptarse a la creciente cantidad\\
    de sensores y dispositivos conectados y proporcionar\\
    una capacidad de procesamiento adecuada para manejar\\
    los grandes volúmenes de datos.
  } \\ \hline

  RNF-FIW7 & Integración con tecnologías de vanguardia
  & \makecell[l]{
    Debe ser compatible con tecnologías emergentes,\\
    como inteligencia artificial o el aprendizaje automático,\\
    para poder ofrecer soluciones de IoT avanzadas.
  } \\ \hline

\end{longtable}
}
\section{Apéndice}
\subsection{Acrónimos y Abreviaturas}
\begin{itemize}
    \item \textbf{IoT}\ttab Internet de las Cosas.
    \item \textbf{API}\ttab Application Program Interface.
  \end{itemize}
\end{document}